\begin{center}
  \thispagestyle{empty}

  \noindent 데이터마이닝(나) 2차 팀 과제 보고서 \hfill \small \makeatletter\texttt{Last Updated: \@date}\makeatother \\
  \noindent\rule{\textwidth}{0.5pt}\par

  \vspace{2em}

  \huge \textbf{청소년기의 심리·정서적 요인을 통한 성인 진입기 진로 안정형·탐색형 성향 분류 예측} %
  \vspace{0.5em}
  \\
  \large \textbf{Predicting Career Stability-Exploration Types through Psychological and Emotional Factors in Adolescence} \\

  \vspace{1em}
  \noindent\rule{\textwidth}{0.5pt}\par

  \vspace{1em}

  \small \textbf{김형훈$^1$, 박소윤$^2$, 이수현$^3$}
  \vspace{0.5em}
  \\
  \small $^1$숭실대학교 산업정보시스템공학과 20192208 \\
  \small $^2$숭실대학교 산업정보시스템공학과 20231342 \\
  \small $^3$숭실대학교 산업정보시스템공학과 20231359

  \vspace{1em}

  \noindent\textbf{Abstract}
  본 연구는 청소년기의 다차원적인 심리·정서적 특성이 성인 진입 초기(만 18-19세)의 진로 이행 패턴에 미치는 예측적 영향을 종단적으로 탐색하고자 하였다. 이를 위해 한국청소년패널조사(KYPS)의 중2 패널 1차년도부터 6차년도까지의 데이터를 활용하였으며, 연구 참여자의 청소년기(중2-고3) 심리·정서적 특성은 요인 분석을 통해 다차원적 요인 점수로 구성되었다.

  성인 초기 진로 상태는 6차년도(만 18-19세) 시점의 활동 유형을 Arnett(2000)의 성인 진입기 이론에 기반하여 ‘안정형’과 ‘탐색형’의 이진 범주로 재분류하여 종속변수로 사용하였다. 패널 데이터의 종단적 특성과 복합 표본 설계를 고려하여, 1차부터 6차년도까지 지속적으로 참여한 패널을 대상으로 6차년도 종단면 가중치를 적용한 머신러닝 분류 분석을 수행하였다. 모델의 예측 성능은 별도의 평가 데이터셋에서 혼동 행렬, 정확도, 정밀도, 재현율, F1-점수 및 ROC AUC를 통해 종합적으로 평가되었다.

  분석 결과, 청소년기 심리·정서적 특성은 만 18-19세 시점의 진로 상태(안정형 vs. 탐색형) 분류에 일정 부분 기여하였으나, 전반적인 예측력은 기대만큼 높지 않았다. 세 모델 모두 무작위 분류기보다는 통계적으로 유의미하게 향상된 성능을 보였으며, 특히 ‘안정형’ 진로 상태에 대한 분류 정확도가 ‘탐색형’에 비해 상대적으로 양호하게 나타났다. 그러나 ‘탐색형’ 진로 유형에 대한 예측, 특히 ROC AUC 기준의 변별력은 제한적인 수준에 머물렀다.

  본 연구는 성인 진입기의 진로 이행 패턴을 청소년기의 심리·정서적 발달과 연결하여 종단적으로 탐색하고, 머신러닝 기법을 활용한 진로 유형 예측 모델 구축의 초기 가능성과 한계를 동시에 제시했다는 점에서 의의를 찾을 수 있다.
  본 연구의 결과는 향후 청소년 진로 상담 및 교육 정책 수립 시 고려할 수 있는 기초자료를 제공하며, 개별 맞춤형 진로 지도 프로그램 개발을 위한 추가적인 논의와 연구방향을 제시하는데 활용될 수 있을 것으로 기대된다.
  \vspace{0.5em}
  \\
  \noindent\textbf{주제어 :} 한국아동·청소년패널조사(KCYPS), 랜덤 포레스트, XGBoost, Logistic Regression, Arnett 성인 진입기 이론, 청소년기 심리 상태, 진로 성향

  \vspace{1em}

\end{center}
