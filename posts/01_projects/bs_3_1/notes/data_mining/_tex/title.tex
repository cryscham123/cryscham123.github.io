\begin{center}
  \thispagestyle{empty}

  \noindent 데이터마이닝(나) 2차 팀 과제 보고서 \hfill \small \makeatletter\texttt{Last Updated: \@date}\makeatother \\
  \noindent\rule{\textwidth}{0.5pt}\par

  \vspace{2em}

  \huge \textbf{청소년기의 심리·정서적 요인을 통한 진로 의사결정 성향 예측} %
  \vspace{0.5em}
  \\
  \large \textbf{The Influence of Adolescent Psychological and Emotional Factors on Career Decision-Making Propensity} \\

  \vspace{1em}
  \noindent\rule{\textwidth}{0.5pt}\par

  \vspace{1em}

  \small \textbf{김형훈$^1$, 박소윤$^2$, 이수현$^3$}
  \vspace{0.5em}
  \\
  \small $^1$숭실대학교 산업정보시스템공학과 20192208 \\
  \small $^2$숭실대학교 산업정보시스템공학과 20231342 \\
  \small $^3$숭실대학교 산업정보시스템공학과 20231359

  \vspace{1em}

  \noindent\textbf{Abstract}
  본 연구는 청소년기 심리 상태가 성인기 초기(만 18-19세) 진로 이행에 미치는 영향을 예측적으로 탐색하고자 하였다. 이를 위해 한국청소년패널조사(KYPS) 중2 패널 1차년도부터 6차년도까지의 종단 데이터를 활용하였다. 연구 참여자의 청소년기 심리 상태는 다차원적인 요인으로 측정되었으며, 성인기 초기 진로 상태는 만 18-19세 시점의 상세 활동 유형으로 정의되었다.

  패널 데이터의 종단적 특성과 복잡한 표본 설계를 반영하기 위해, 1차년도부터 6차년도까지 모두 응답한 패널을 대상으로 하는 6차년도 종단면 가중치를 적용한 가중치 고려 분류 분석을 수행하였다. 종속 변수인 만 18-19세 진로 상태는 원 데이터의 상세 분류 코드를 연구 목적과 데이터의 예측 가능성을 고려하여 '경제 활동 참여 상태 (취업, 창업, 가업 등)'와 '기타 상태 (대학 재학/준비, 비활동, 학업 지속 등)'의 이진 범주로 재범주화하였다. 독립 변수로는 청소년기 심리 상태 요인 점수를 활용하였다. 가중 랜덤 포레스트 모델을 훈련시키고, 별도의 평가 데이터셋에서 가중 혼동 행렬 및 가중 정확도, 가중 정밀도, 가중 재현율, 가중 F1-score 등 가중 성능 지표를 산출하여 모델의 예측 성능을 평가하였다.

  분석 결과, 청소년기 심리 상태 변수는 만 18-19세 시점의 진로 상태를 '생산적 활동 참여 상태'와 '비경제활동 및 학업 관련 상태'로 구분하는 데 매우 강력한 예측력을 보였다.

  본 연구는 청소년기 심리 상태가 성인기 초기 진로 선택 중 경제 활동 참여 여부를 예측하는 데 중요한 역할을 함을 실증적으로 밝혔다는 의의를 가진다. 이러한 결과는 경제 활동 비참여 고위험 청소년을 조기에 식별하고 맞춤형 진로 및 심리 지원 정책을 개발하는 데 활용될 수 있다.
  \vspace{0.5em}
  \\
  \noindent\textbf{주제어 :} 한국아동·청소년패널조사(KCYPS), 랜덤 포레스트, 청소년기 심리 상태, 조기 경제 활동 참여

  \vspace{1em}

\end{center}
