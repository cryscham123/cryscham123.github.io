\begin{center}
  \thispagestyle{empty}

  \noindent 데이터마이닝(나) 2차 팀 과제 보고서 \hfill \small \makeatletter\texttt{Last Updated: \@date}\makeatother \\
  \noindent\rule{\textwidth}{0.5pt}\par

  \vspace{2em}

  \huge \textbf{청소년기의 심리·정서적 요인을 통한 성인 진입기 진로 안정형·탐색형 성향 분류 예측} %
  \vspace{0.5em}
  \\
  \large \textbf{Predicting Career Stability-Exploration Types through Psychological and Emotional Factors in Adolescence} \\

  \vspace{1em}
  \noindent\rule{\textwidth}{0.5pt}\par

  \vspace{1em}

  \small \textbf{김형훈$^1$, 박소윤$^2$, 이수현$^3$}
  \vspace{0.5em}
  \\
  \small $^1$숭실대학교 산업정보시스템공학과 20192208 \\
  \small $^2$숭실대학교 산업정보시스템공학과 20231342 \\
  \small $^3$숭실대학교 산업정보시스템공학과 20231359

  \vspace{1em}

  \noindent\textbf{Abstract}
  본 연구는 청소년기 심리·정서적 특성이 성인기 초기(만 18-19세) 진로 이행 패턴에 미치는 영향을 예측적으로 탐색하고자 하였다. 이를 위해 한국청소년패널조사(KYPS) 중2 패널 1차년도부터 6차년도까지의 종단 데이터를 활용하였다. 연구 참여자의 청소년기 심리·정서적 특성은 다차원적 요인으로 측정되었으며, 성인기 초기 진로 상태는 만 18-19세 시점의 상세 활동 유형으로 정의되었다.

  패널 데이터의 종단적 특성과 복잡한 표본 설계를 반영하기 위해, 1차년도부터 6차년도까지 모두 응답한 패널을 대상으로 6차년도 종단면 가중치를 적용한 가중치 고려 분류 분석을 수행하였다. 종속 변수인 만 18-19세 진로 상태는 Arnett(2000)의 성인 진입기 이론에 기반하여 '안정형'과 '탐색형'의 이진 범주로 재범주화하였다. 독립 변수로는 청소년기 심리·정서적 특성 요인 점수를 활용하였다. 랜덤 포레스트, XGBoost, 로지스틱 회귀분석 모델을 훈련시키고, 별도의 평가 데이터셋에서 혼동 행렬 및 정확도, 정밀도, 재현율, F1-score 등 성능 지표를 산출하여 모델의 예측 성능을 평가하였다.

  분석 결과, 청소년기 심리·정서적 특성은 만 18-19세 시점의 진로 상태(안정형 vs. 탐색형) 분류에 기여하였으나, 그 예측력은 전반적으로 높지 않았다. 세 모델 모두 무작위 분류기보다는 우수한 성능을 보였으며, '안정형' 진로 상태에 대한 분류 정확도가 '탐색형'에 비해 상대적으로 양호하였다.

  본 연구는 성인 진입기의 진로 이행 패턴을 청소년기 심리·정서적 발달과 연결하여 종단적으로 탐색하고, 머신러닝 기법을 활용한 진로 유형 예측 모델 구축의 초기 가능성을 제시했다는 점에서 의의를 찾을 수 있다. 본 연구의 결과는 향후 청소년 진로 상담 및 교육 정책 수립 시 고려할 수 있는 기초 자료를 제공하며, 개별 맞춤형 진로 지도 프로그램 개발을 위한 추가적인 논의와 연구 방향을 제시하는 데 활용될 수 있을 것으로 기대된다.
  \vspace{0.5em}
  \\
  \noindent\textbf{주제어 :} 한국아동·청소년패널조사(KCYPS), 랜덤 포레스트, XGBoost, Logistic Regression, Arnett 성인 진입기 이론, 청소년기 심리 상태, 진로 성향

  \vspace{1em}

\end{center}
